\documentclass[a4paper,12pt]{article}
\usepackage{caption}
\usepackage{subcaption}
\usepackage{fancyhdr}
\pagestyle{fancy}
\lhead{tqvj24}
\chead{Written Answers}
\rhead{Computer Graphics}

\begin{document}

\section*{Question 1}
\subsection*{a)}
% Difference between attribute, uniform and varying in shaders
\begin{description}
    \item[Uniform]
    A global variable that cannot be changed by the shaders, and is dependent upon the geometric primitive in use.
    % per primitive, basically doesn't change
    \item[Attribute]
    A global variable that cannot be changed by the shader, which is dependent upon the current vertex, and therefore is only available within the vertex shader.
    % per vertex
    \item[Varying]
    A variable that stores data that has been interpolated, so can vary pixel by pixel. The vertex shader is able to modify this, however the fragment shader is not.
    % interpolating (fragments)
\end{description}

\subsection*{b)}
\begin{verbatim}
var polygonVerticesAndColurs = new Float32Array([
     0.0,  1.0,  -4.0,  0.4,  1.0,  0.5,
    -0.5, -1.0,  -4.0,  0.4,  0.7,  0.4,
     0.5, -1.0,  -4.0,  0.5,  0.4,  0.5,
     0.0,  1.0,  -2.0,  1.0,  1.0,  0.4,
    -0.5, -1.0,  -2.0,  0.3,  1.0,  0.5,
     0.5, -1.0,  -2.0,  1.0,  0.4,  0.4
]);
var n = 6; // 6 vertices in this polygon

// Create a buffer object
var vertexColorbuffer = gl.createBuffer();
if (!vertexColorbuffer) {
  console.log('Failed to create the buffer object');
  return -1;
}

// Write the vertex information and enable it
gl.bindBuffer(gl.ARRAY_BUFFER, vertexColorbuffer);
gl.bufferData(gl.ARRAY_BUFFER, polygonVerticesAndColurs, gl.STATIC_DRAW);
\end{verbatim}
Below is further code relating to using this array, as I was unsure how much code needed to be included.\newpage
\begin{verbatim}
var FSIZE = verticesColors.BYTES_PER_ELEMENT;

// Assign the buffer object to a_Position and enable the assignment
var a_Position = gl.getAttribLocation(gl.program, 'a_Position');
if(a_Position < 0) {
  console.log('Failed to get the storage location of a_Position');
  return -1;
}

gl.vertexAttribPointer(a_Position, 3, gl.FLOAT, false, FSIZE * 6, 0);
gl.enableVertexAttribArray(a_Position);

// Assign the buffer object to a_Color and enable the assignment
var a_Color = gl.getAttribLocation(gl.program, 'a_Color');
if(a_Color < 0) {
  console.log('Failed to get the storage location of a_Color');
  return -1;
}
gl.vertexAttribPointer(a_Color, 3, gl.FLOAT, false, FSIZE * 6, FSIZE * 3);
gl.enableVertexAttribArray(a_Color);
\end{verbatim}

\subsection*{c)}
TODO SCENE GRAPH

\subsection*{d) TODO}
% (Reference: WebGL Prog. Guide Chapter 4, Table 4.1) Suppose drawBox(m) is a
% function to draw a transformed box according to the transformation matrix m. That is, if m is
% a rotation matrix, the function will draw a rotated box.
\subsubsection*{i.}
% Explain the meaning of the following code segment and state the result obtained:
% m.setRotate(angle, 0.0, 1.0, 0.0);
% m.translate(1.0, 3.0, -5.0);
% drawBox(m);
% [5 marks]
\subsubsection*{ii.}
% ii) Explain whether you will get the same result if m.setRotate() has been replaced by
% m.rotate().

\section*{Question 2}
The included files \texttt{tqvj24.html} and \texttt{tqvj24.js} are my code for this question.

\end{document}
